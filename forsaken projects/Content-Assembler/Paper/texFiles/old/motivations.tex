\section{Motivations}\label{mindMappingUse}
	For the web site configuration by end users, we have choosen a visual representation
	which is the mind mapping. And this for three reasons :
	
	\textbf{Complexe WOCL metamodel simplification :} By comparing
	the WOCL metamodel(see paragraph \ref{abstractSyntax}) and a mind map example from the mind mapping tool (paragraph \ref{concreteSyntax}), it is easy to see this kind of visual presentation decreases the complexity of 	our WordPress DSL.
		
	\textbf{More simple than the use of the WordPress Dashboard :} Indeed, with the WordPress dashboard, it is necessary to naviguate in the various menus to manage an element, but with the mind-mapping, it is just a node addition / modification. Furthermore, the end users need different technical knowledges to  configure directly this dashboard, contrary to a mind-map where the meta-concepts (nodes, edges, ...) are simples and identics for each website elements representation. Finally, about the default functionalities, there are not present in WordPress : users must have to find a plugin and add it via the dashboard. It is possible to represent default functionalities on the mind-map as presented in the following example : to have a multilanguage plugin (functionality) using the dashboard, you must be connected with a user which has an administrator role, navigate to the menu 'Plugins', select the 'Add New' menu and upload it. With the mind-map, you have just to create a 'Website' node, link it to another node 'Functionality', choose a child node between those by default (multilanguage, seo, indexing, eCommerce, ...)and add it and run the execution (see section \ref{techUse}), without any plugin downloading on the WordPress server.
	
	\textbf{Concepts clearly visibles :} Due to the structure of a mind map (as it is possible to see in the paragraph \ref{representationTools}), concepts are easy to understand, reducing the learning cost of WOCL (because mind mapping is more natural\cite{mindMapping}).		