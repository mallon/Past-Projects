\section{Related work}\label{relatedWorks}
The use of a DSL to simplify the web creation for end users was inspired by the
work of O.Diaz and G.Puente. In this paper, they present their own DSL - 'WSL' - for the creation of wiki via the mind mapping and the MediaWiki platform\cite{wikiScafolding}.

One other work\cite{conceptMod} focus on the data conceptual modelisation and communication to end users, using respectively two kinds of specific representation : an XML Document type Definition (DTD) and an XML Document Object Model (DOM).

Concerning a simpler (graphical) development support for more advanced end users (or developers), some other paper proposed a web interface, where the programming and the GUI creation processes are not separated\cite{livelyFabrik}. Their work is based on the Lively Kernel\cite{lively} of the Sun Labs.

Other works\cite{wsol,dslMashups} focusing more on functionalities, and more precisely on the proposed web services : they have made DSLs to improve the web services management, their compositions, and their creations.

Another tool \cite{dictator} enables to configure WordPress, but it is limited: indeed, it is technically only for the WordPress CMS (contrary to our tool, which could be easily applied to other CMS like Joomla or Drupal), and it uses a textual representation, which does not allow to have an equivalent quick and comprehensive overview of items added or changed.

Finally, a design tool using\cite{webRatio} a specific DSL was also proposed : with its specific DSL 'WebML', the tool allows designing complex Web and SOA\footnote{Service Oriented Architectures} applications, by providing visual design facilities and code generation. Because they used for the modeling step a representation which is less comprehensive for end users than a mind map and also because the textual representation make easier the web designer work, our tool and concepts could be integrate in it.