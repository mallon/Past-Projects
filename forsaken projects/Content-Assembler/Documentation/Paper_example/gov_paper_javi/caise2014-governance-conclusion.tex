In this paper we have presented an approach that opens the possibility to explicitly define and automatically aply the rules that govern a software development project. Our approach includes a new DSL for specifying the rules, a collaboration infrastructure to record the interactions (and to be able to trace them back) and a decision engine to automatically update the tasks status based on that. A prototype tool implemented as an Eclipse plug-in and relying on the Mylyn component for the connection with the most popular tracking systems has been developed. Note that although our approach supports both the definition and application of governance rules, they can be used independently and project leaders can therefore decide using only the former, which would improve the transparency and sustainability of governance model.

%As a direct application of our approach, we plan to use it to improve our DSL itself. By using a tracking system, users can request language improvements (e.g., including new constructs, modifying keywords, etc.) that are then voted and decided according to governance rules which have been defined explicitly\footnote{The project is located at http://}. 

As further work, we would like to provide support for evolution in governance rules as they depend on social interactions it is common to find temporal exception, ambiguities and changes over time.
%extend our rule definition language to be able to specify as well the rules governing the team organization (e.g., how the project participants can be promoted between the different roles). 
Adding privacy concerns is also under evaluation (some projects may require anonimity in the votation phase, or keep private some discussions to all people with less privileges). Finally, we would like to mine existing software repositories to infer and study the governance rules they are using. It would be interesting to see if the extracted rules correspond to the perception of (external) participants (e.g., whether they have evolved through time or whether their application is full of exceptions) and how they correlate with other metrics of the projects.
