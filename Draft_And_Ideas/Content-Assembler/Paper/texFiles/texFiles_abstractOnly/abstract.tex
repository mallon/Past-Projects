\begin{abstract}
%One short sentence describing state of the art
The websites development was simplified over time, notably via
the Web standardization, but also through the use of CMS \footnote{Content Management System}.\\
%One / two short sentences desc. drawbacks
However, the use of the various available tools still requires technical knowledge, that web designers have (concerning CMS itself, web standards, or databases), unlike many end users (which can be an obstacle to the web site creation).\\
%One short sentence desc. conceptual solution(s) presented in the paper
In this paper, we propose a simplification of the website creation for end users and, for web designers, an improvement of the configuration of one or several sites.\\
%One - Three sentences desc. advantages of choosen conceptual solutions
Allowing end users to create their websites will enable them to implement their ideas directly, without being slowed by technical points, but also to reduce the gap between concepts and practice. Simplification for web designers will let them to configure and manage easier and quicker their  website(s).\\
% One - three sentences desc. implementation of choosen conceptual solutions
So we create a DSL\footnote{Domain Specific Language}~\cite{dsl} and its metamodel~\cite{metaExplanation} to represent the general structure of a website configured via the WordPress CMS (which is the most used one currently~\cite{cmsRepartition}). By using the DSL metamodel and the mind mapping (via a software~\cite{freeplane} extension), we enabled end users to create their site simpler.\\
For web designers, we integrate the DSL in an IDE\footnote{Integrated Development Environment}, via an Eclipse product\cite{eclipseProduct} and the use of the XText Eclipse plugin~\cite{xtext}, to have a configuration file for one or several websites (and therefore to configure them remotely).\\
\keywords{Website developement, CMS, DSL, Mind mapping}
\end{abstract}