\section{Conclusion}
In this paper we have presented an approach allowing a simpler web site configuration for users. \textit{For this web site configuration}, we have used the \textit{WordPress CMS}, and \textit{our new DSL ('CMS-CL' : CMS Configuration Language)} to represent the WordPress configuration. Then, for the end users, we have used our DSL on top of the \textit{mind-mapping} to allow them to manipulate the different elements and easily configure the web site. About the web designers, we have created an IDE (using an Eclipse product and the XText plugin) using the DSL to textually defined the web site configuration.

As further work, it would be interesting to provide the possibility for end users to \textit{use our DSL in the same way, but collaboratively} : the DSL would be manuipulated online, and each authorized user could be able to manage web site elements, communicating with other users. In addition, \textit{this collaboration} between users \textit{could be apply on several sites}, i.e. a user could access to several online representations of different websites. The benefits given by this collaborative work platform would be the quality and speed improvement of the web sites creation. Another interesting thing, would be \textit{to create a more generalized DSL to use it with other CMS}.