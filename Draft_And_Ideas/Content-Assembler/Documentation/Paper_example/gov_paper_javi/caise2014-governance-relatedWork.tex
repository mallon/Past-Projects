The study of how people coordinate to develop a software system has been a research topic for a long time~\cite{Herbsleb1999,Kraut1995,Parnas1972,Crowston2005,Markus2007}, including the broader topic of IT governance \cite{Chulani2008,Ramasubbu2008,sg,Webb2006}. In the field of software governance in OSS, several works have tried to classify how these projects are managed \cite{Laat2007,Markus2007,DeNoni2013}. However, to the best of our knowledge, little attention has been paid to the precise definition and support for the governance rules in specific projects.

Some works report on concrete case studies. For instance, \cite{Aranda2009} reports on the coordination activities performed to solve bugs (e.g., how bugs are detected and closed, means used to coordinate the work, etc.). In \cite{Guzzi2013} authors present how developers use mailing list during the development of FLOSS systems (e.g., which are the main topics discussed or the level or participation of the different user roles). In \cite{osswatch} some indications are provided to describe governance models in natural language.

Other approaches focus on leveraging collaboration information obtained by mining software repositories to infer coordination relationships and structures.  The work presented in \cite{Bird2008} describes an approach to discover potential social structures from the threads created in the mailing list of the project. In \cite{Heller2011} visualization techniques are applied to easily discover communication patterns from Github repository metadata (e.g., the effect of geographic distance among developers, influence among cities, etc.). The tool called CrowdWeaver is present in \cite{Kittur2012}, which allows coping with the complexity of managing crowdsourced projects. These tools could be adapted to facilitate the discovery of governance rules using our approach.

Concrete methodologies for collaboration strategies have also been proposed. For instance, \cite{Duque2012} and \cite{Luther2013} present approaches to support collaborative processes in groupware systems and online creative projects, respectively. However, they do not provide mechanisms to define and apply the governance rules to apply in each project.

Collaboration has also been the focus on two other DSLs. The approach presented in \cite{Gallardo2012} describes a DSL to represent collaboration workflows that can appea in modeling tools (e.g., the steps needed to create a class diagram when several users are collaborating). In our previous work \cite{bib:collaboro} we presented a DSL-based tool to collaborativelly develop DSLs, thus allowing representing the collaborations arisen in the process. Again, they do not provide any support for the governance of such collaborations.

