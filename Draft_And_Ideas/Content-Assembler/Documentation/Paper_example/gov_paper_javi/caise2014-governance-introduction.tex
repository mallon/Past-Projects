The development of large-scale software has to cope with a huge number of tasks consisting of either implementing new issues or fixing bugs~\cite{Aranda2009}. Thus, effective and precise prioritization of these tasks is key for the success of the project. For this purpose, each project defines and applies its own set of governance rules, which also define how to contribute in the project and how decisions are made (e.g., which issues have to be implemented or when the next milestone will be). Thus, a governance model enables the coordination of developers in order to advance the project. According to Conway's Law~\cite{Conway1968}, this coordination has a direct impact on the product being built \cite{Herbsleb1999}.  

Governance models makes particularly sense in the context of Open Source System (OSS) development projects \cite{osswatch}, where external developers can participate actively and need to know how they can contribute. However, these projects usually have a methodology very loosely defined (if any)~\cite{McConnell1999} and it is hard to find an explicit system-level design, a project plan, schedule or list of deliverables~\cite{Mockus}. Thus, despite their importance, in practice governance models are hardly ever explicitly defined. This hampers the integration of new developers\footnote{E.g., searching for \emph{how to contribute} in \emph{Stackoverflow} returns more than four thousand results (i.e., $\sim$8\% of the total number of questions)} in the team who must spend some time understanding the ``culture'' of the project \cite{Shibuya2009}. 

%For this purpose, each project defines and applies its own set of governance rules. These rules can be as simple as \emph{team leaders decide the tasks to do} or more complex as \emph{the task to be done will be the one most voted by the developers participating in the project}. Governance rules enable the coordination of developers in order to advance the project. According to Conway's Law~\cite{Conway1968}, this coordination has a direct impact on the product being built \cite{Herbsleb1999}.  

%Despite their importance, in practice governance rules are hardly ever explicitly defined, specially in the context of Open Source Systems (OSS), where the development methodology is very loosely defined (if any)~\cite{McConnell1999} and it is hard to find an explicit system-level design, a project plan, schedule or list of deliverables~\cite{Mockus}. Moreover, the support for enforcing such governance rules is even more limited. This hampers the integration of new developers in the team who must spend some time understanding the ``culture'' of the project \cite{Shibuya2009}. 

The governance model being missing, mechanisms to facilitate the communication and the assignment of work are considered crucial for the success of the development~\cite{Mockus2000,Crowston2012}. Tracking systems (i.e., bug-traking such as Bugzilla\footnote{http://www.bugzilla.org/} and issue-tracking systems such as Mantis\footnote{http://www.mantisbt.org/}) are broadly used to manage the tasks to be performed. Other collaborative tools such as mailing-lists or forums are also used to coordinate the developers involved in the project. While these tools provide a convenient compartmentalization of work and effective means of communication, they fall short in providing adequate support for specifying and applying a governance model (e.g., supporting the voting of tasks, easy tracking of decisions made in the project, etc.). 

Therefore, the explicit definition of a governance model would have several benefits, including improvements in the transparency and sustainability \cite{osswatch}. Moreover, we believe the support for not only the definition but also the application of the governance model would also promote traceability (being able to track why a decision was made and who decided it) and the automation of the governance process (e.g., liberating developers from having to be aware and follow the rules manually, minimizing the risk of inconsitent behaviour in the evolution of the project). 

%Therefore, we believe the explicit definition of a governance model along with the corresponding infrastructure to help developers realize it would have several benefits, including improvements in the transparency of the decision-making process, traceability (being able to track why a decision was made and who decided it) and the automation of the governance process (e.g., liberating developers from having to be aware and follow the rules manually, minimizing the risk of inconsitent behaviour in the evolution of the project). 

This paper tackles both aspects by providing a new domain-specific language (DSL) to let project managers easily define the governance rules of their projects and a collaborative infrastructure to be able to ``execute'' those rules as part of the project evolution, though the use of the latter is up to the project managers. The DSL is defined based on our analysis of the (implicit) rules used in the governance of a set of OSS development projects. The collaborative infrastructure has been implemented on top of Mylyn, a project management plugin for the Eclipse platform with connectors for most popular tracking systems. Thanks to these connectors our approach can be used together with the existing tracking systems already in place in software development projects. 

The rest of the paper is structured as follows. In the next section, we report on the lack of explicit support for defining governance models in current OSS development projects and the various governance strategies used therein. We then present our proposal consisting in a DSL to describe these rules, the infrastructure needed to apply them and an illustrative case study. Finally, we describe our tool support, compare our approach with related work and draw some final conclusions and future work.


